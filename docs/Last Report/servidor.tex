\chapter{Servidor}
\label{chap.servidor}

O programa server.py for construído  com o módulo CherryPy \cite{cherry}. Fizemos as seguintes importações: cherrypy, sqlite3 e os.path. As restantes importações são programas construídos pelos membros do grupo: interpreter, synthesizer, effects\_processor e database.

O server.py possui uma única class (Root) onde se encontram os seguintes módulo criados: mobile, addVote, delVote, index, novainterpretacao, tocarmusica, sobrenos, createSong, createInterpretation, getWaveForm, getWaveFile, getNotes, listSongFiles, listNotes, listSongs.

\section{index, novamusica, novainterpretacao, tocarmusica}
Verifica se o website esta a ser acedido através de um dispositivo móvel ou computador. Caso seja móvel irá redirecionar sempre para o index da versão mobile. Caso contrário, devolve a página pedida na versão desktop.

\section{createSong}
Recebe como argumentos nome e pauta, verificando se estão minimamente válidos (nome não pode ser vazio, a pauta não pode estar vazia - quando javascript envia notas vazias, o módulo recebe-as no formato undefined:undefined). Após a verificação é feita decodificação para String dos parâmetros recebidos. Depois de converter é criada a imagem com o nome igual ao ID das notas no tabela musics da base de dados. A criação de imagem também serve de metodo de verificação, pois se ocorrer algum erro na criação, uma excessão é lançada e o programa envia mensagem de erro. Caso a imagem seja criada com sucesso, os dados são enviados para base de dadose e é devolvida uma mensagem de sucesso.

\section{createInterpretation}
Recebe como argumentos, registo, id da pauta, efeito e nome, fazendo a verificação (registo tem de ser númerico, tamanho do registo só pode ser 9, o registo não pode conter o algarismo 9, nome não pode ser vazio, efeito não pode ser vazio, id não pode ser vazio, id tem de ser numerico). Antes de criar o ficheiro .wav é necessário ir buscar e juntar o nome e as notas do id fornecido como argumento. Caso na criação do ficheiro .wav ocorra uma excessão, o programa imprime mensagem de erro. Se o ficheiro for criado com sucesso, a informação sobre a interpretação (argumentos do módulo) irá ser guardada na base de dados e irá ser devolvida uma mensagem de sucesso.

\section{getWaveForm, getWaveFile}
Recebe como argumento o id da música/interpretação, que depois irá redirecionar para o ficheiro .jpg/.wav. Caso o ficheiro nao exista será lançada a excessão 404NotFound.

\section{getNotes}
Recebe como argumento o id da tabela musics devolvendo as notas na forma de String.

\section{listSongFiles}
Recebe como argumento o id da tabela musics, onde irá buscar todas as interpretações feitas sobre essa musica, retornando na forma de json.

\section{listNotes, listSongs}
Não recebe nenhum argumento, devolvendo em formato json todas as tabelas de musics(tablema das notas) / interpretations(table das interpretações), respetivamente.

Formato do json devolvido (listNotes):
\begin{lstlisting}
[
	{
		"notes": "d=4,o=5,b=160:c.6, e6, f#6, 8a6, g.6, e6, c6, 8a, 8f#, 8f#, 8f#, 2g, 8p, 8p, 8f#, 8f#, 8f#, 8g, a#., 8c6, 8c6, 8c6, c6",
		"id": 1,
		"name": "The Simpsons"
	}
]
\end{lstlisting}
Formato do json devolvido (listSongs) : 

\begin{lstlisting}
[
	{
		"name": "The Simpsons Rock Style",
		"id_music": 1,
		"downvotes": 10,
		"effects": "distortion",
		"registration": 888800000,
		"upvotes": 2,
		"id": 1
	}
]
\end{lstlisting}

\section{addVote, delVote}
Recebe como argumento o id da interpretação onde irá adicionar, na tabela interpretations, uma unidade ao campo upvote / downvote, respetivamente. Caso a interpretação não exista ou a operação não ocorra por outro motivo, devolve mensagem de erro.

\section{mobil}
(Para testes) Força a aparecer a versão mobile do site, mesmo que o utilizador esteja a aceder a partir de um computador.